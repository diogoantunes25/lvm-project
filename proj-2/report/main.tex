\documentclass[12pt]{article}

\usepackage[margin=1in]{geometry}
\usepackage{amsmath,amsthm,amssymb}

\usepackage{graphicx}
\usepackage{enumitem}

\usepackage{algpseudocode}
\usepackage{algorithm}

\usepackage{tikz}

\usepackage{xcolor}
\definecolor{darkgreen}{rgb}{0.0, 0.4, 0.0}
\definecolor{darkyellow}{rgb}{1.0, 0.5, 0.0}

\newcommand{\wip}{\textbf{(WIP) }}
\newcommand{\tba}{\textbf{(TBA) }}


\newcommand{\idea}[1]{\textcolor{darkyellow}{#1}}
\newcommand{\drafter}[1]{\textcolor{darkgreen}{#1}}
\newcommand{\draft}[1]{\textcolor{purple}{#1}}
\newcommand{\sad}[1]{\textcolor{red}{#1}}
\newcommand{\blah}{\textbf{blah blah blah}}

\newcommand{\dsa}[1]{\textbf{[DSA: #1]}}

\begin{document}

\title{Project 2 — Report}
\author{
  Diogo Antunes\\
  99210
  \and
  Javier María\\
  99240
  \and
  Tomás Silva\\
  98973
}

\maketitle

\section*{Exercise 1}

\subsection*{Exercise 1.1}

\subsubsection*{Exercise 1.1 (a)}

\idea{The justification is as expectd -- the safisfiable formulas is proven with \blah{}.}
\idea{The unsatisfiable formulas are proven by providing a trace of the transition system that does not satisfy the formula.}

\begin{enumerate}[label=\roman*.]
  \item $\mathsf{F}~\mathsf{G}~c$ -- not satisfied

  \idea{The trace is $s_2 (s_4 s_3)^\omega$.}

  \item $\mathsf{G}~\mathsf{F}~c$ -- satisfied

  \blah{}

  \item $(\mathsf{X}\neg~c) \rightarrow (\mathsf{X}~\mathsf{X}~c)$ -- satisfied

  \blah{}

  \item $\mathsf{G}~a$ -- not satified

  \idea{The trace is $s_1 s_4 s_5^{\omega}$.}

  \item $a~\mathsf{U}~\mathsf{G}(b \vee c)$ -- satisfied

  \blah{}

\end{itemize}

% TODO: add proofs / counter example

\subsubsection*{Exercise 1.1 (b)}

\idea{The modeling of the system in Promela was not trivial for two reasons -- modeling the possibility of multiple initial states and avoiding the effect of stutter steps in the use of the next operator in $c$.}
\idea{The first problem was solved by adding a flag to signal the start of the system.}
\idea{To ensure that the LTL formulas refered to the inital state of the transition system being modelled and not Promela, the formula $\Phi$ is modified to $\neg started~\mathsf{U}~(started~\wedge~\Phi).$}
\idea{Additionally, the use of stutter-steps by Promela also posed problems when trying to model things like using the next operator.}
\idea{This is because next operator of Promela won't match the next operator of the state machine being.}
\idea{To solve this a flag that is toggled on every state transition and to the next only applies if the flag is toggled.}
\idea{With this rewrite, the LTL formulas are encoded in the following way.}

\idea{The last bullet showed that i. and iv. are not satisfied and Spin confirms this.}
\idea{The traces are the following. The complete are provided in a separate text file \blah{}.}

\subsection*{Exercise 1.2}

% TODO: add proofs / counter example

\section*{Exercise 2}

\subsection*{Exercise 2.1}

\idea{The encoding of properties provided is the following (this is also in file \texttt{road\_ltl.pml}):}

\begin{enumerate}[label=(\alph*)]
  \item $\mathsf{G}\neg(\texttt{W\_LIGHT} \wedge \texttt{E\_LIGHT})$

  \item $\mathsf{G}(\texttt{W\_LIGHT} \rightarrow \mathsf{F}~\neg\texttt{W\_LIGHT} \wedge \texttt{E\_LIGHT} \rightarrow \mathsf{F}~\neg\texttt{E\_LIGHT})$

  \item $\mathsf{G}(\texttt{on\_lane} \rightarrow \mathsf{F}\neg\texttt{on\_lane})$

  \item $\mathsf{G}((\texttt{W\_LIGHT} \wedge \texttt{on\_lane}) \rightarrow \neg\texttt{E\_LIGHT})$

  \item $\mathsf{G}(\texttt{at\_w} \rightarrow \mathsf{F}~\texttt{W\_LIGHT})$
\end{enumerate}

\subsection*{Exercise 2.2}

\idea{A safety property is \blah{} and a liveness properti is \blah{}.}
\idea{A property is said to be a combination when \blah{}.}
\idea{With this the formulas above can be classified in the following manner.}

% FIXME: this definitely needs to be revisited

\begin{enumerate}[label=(\alph*)]
  \item Safety property

  \item Liveness property

  \item Liveness property

  \item Safety property

  \item Combination

\end{enumerate}

\subsection*{Exercise 2.3}

\idea{The counter example provided by Spin is the following (the detailed text file is provideed as well).}

% TODO: paste counter example

\subsection*{Exercise 2.4}

\idea{The initial model fails to satisfy property (e) because it does not force \blah{}.}
\idea{That is fixed by \blah{}.}

\section*{Exercise 3}

\subsection*{Exercise 3.1}

% TODO: explain main modelling (particularly the main tricks on byz)

\subsection*{Exercise 3.2}

% TODO: formalization of properties

\subsection*{Exercise 3.3}

% TODO: hand output

\subsection*{Exercise 3.4}

% TODO: hand output

\end{document}
