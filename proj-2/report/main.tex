\documentclass[12pt]{article}

\usepackage[margin=1in]{geometry}
\usepackage{amsmath,amsthm,amssymb}


\usepackage{graphicx}
\usepackage{enumitem}

\usepackage{algpseudocode}
\usepackage{algorithm}

\usepackage{tikz}
\usetikzlibrary{automata, positioning, arrow}

\tikzset{
  ->, % makes the edges directed
  >=stealth, % makes the arrow heads bold
  node distance=3cm, % specifies the minimum distance between two nodes. Change if necessary.
  every state/.style={thick, fill=gray!10}, % sets the properties for each ’state’ node
  initial text=$ $, % sets the text that appears on the start arrow
  }

\usepackage{xcolor}
\definecolor{darkgreen}{rgb}{0.0, 0.4, 0.0}
\definecolor{darkyellow}{rgb}{1.0, 0.5, 0.0}

\newcommand{\wip}{\textbf{(WIP) }}
\newcommand{\tba}{\textbf{(TBA) }}


\newcommand{\idea}[1]{\textcolor{darkyellow}{#1}}
\newcommand{\drafter}[1]{\textcolor{darkgreen}{#1}}
\newcommand{\draft}[1]{\textcolor{purple}{#1}}
\newcommand{\sad}[1]{\textcolor{red}{#1}}
\newcommand{\blah}{\textbf{blah blah blah}}

\newcommand{\dsa}[1]{\textbf{[DSA: #1]}}

\begin{document}

\title{Project 2 — Report}
\author{
  Diogo Antunes\\
  99210
  \and
  Javier María\\
  99240
  \and
  Tomás Silva\\
  98973
}

\maketitle

\section*{Exercise 1}

\subsection*{Exercise 1.1}

\subsubsection*{Exercise 1.1 (a)}

\drafter{The transition system satisfies a formula $\phi$ if all the traces of the transition system are in the models of $\phi$.}
\drafter{For this reason, to prove that the transition system does not satisfy the formula, it is enough to provide a path of the transition system whose trace does not satisfy the formula.}
\drafter{To prove that the transitition system does satisfy the formula, we will argue that any path of the transition system is such that its trace satisfied the formula.}

\begin{enumerate}[label=\roman*.]
  \item $\mathsf{F}~\mathsf{G}~c$ -- not satisfied

  \drafter{The formula states that eventually $c$ will hold forever.}
  \drafter{If execution starts in state $s_2$, moves to $s_4$ and then moves back and forth between $s_3$ and $s_4$, $c$ doesn't always hold given that $c \notin L(s_4)$.}
  \drafter{Formally, the path is $s_2 (s_4 s_3)^\omega$.}

  \item $\mathsf{G}~\mathsf{F}~c$ -- satisfied

  \drafter{The formula states that it's always the case that $c$ will eventually hold.}
  \drafter{The system will start at $s_1$ or $s_2$ and them move to either $s_2$, $s_3$ or or $s_5$, where $c$ holds.}
  \drafter{From these it will go back to $s_4$ and then back to these forever.}
  \drafter{For this reason, no matter the choice of actions, $c$ wil always eventually yhold.}

  \item $(\mathsf{X}\neg~c) \rightarrow (\mathsf{X}~\mathsf{X}~c)$ -- satisfied

  \drafter{The formula states that if in the second state $c$ doesn't hold, then in the third state $c$ will hold.}
  \drafter{There are only two possible second states - $s_3$ and $s_4$.}
  \drafter{Given that $c$ holds at $s_3$ we need only to consider $s_4$.}
  \drafter{In all successors of $s_4$, $c$ holds, which means that the formula is satisfied.}

  \item $\mathsf{G}~a$ -- not satified

  \drafter{The formula states that $a$ always holds.}
  \drafter{This is clearly not the case given that $s_2$ might be the initial state.}
  \drafter{A trace is $s_1 s_4 s_5^{\omega}$.}

  \item $a~\mathsf{U}~\mathsf{G}(b \vee c)$ -- satisfied

  \drafter{This formula states that $G(b \vee c)$ will hold and until then $a$ holds.}
  \drafter{$\mathsf{G}(b \vee c)$ means that at all instants either $b$ or $c$ holds.}
  \drafter{Given the description of the system made previously and the fact the $b \vee c$ holds at $s_2$, $s_3$, $s_4$ and $s_5$, as soon as the system reaches one of these states $b \vee c$ will always hold, which means that $\mathsf{G}(b \vee c)$ holds as well.}
  \drafter{The only other state is $s_1$, where $a$ holds, which means that before reaching $s_2$, $s_3$, $s_4$ and $s_5$ $a$ will hold, which proves that the system satisfies the formula.}

\end{itemize}

\subsubsection*{Exercise 1.1 (b)}

% TODO:

% \drafter{Even thought that Promela / Spin also have an underlying transition system, modelling the modelling of this transition system in Spin is not trivial because there's not a one to one match between the two.}
% \drafter{In particular, one mismatch is that Promela has a single initial state whereas this transition system migth have two.}
% \drafter{Another problem was the fact that Promela might take stutter steps, which render the built-in $\mathsf{X}$ operator hard to use as it won't match the transitions system., a problem that is mention in Promela's reference manual.}
%
% \drafter{The first problem was solved by adding a flag \texttt{started} to signal the start of the system.}
% \drafter{To ensure that the LTL formulas refered to the inital state of the transition system being modelled and not Promela, the formula $\Phi$ about the original transition system is modified to $\neg started~\mathsf{U}~(started~\wedge~\Phi)$ in Promela -- for $\Phi$ to hold in the transition system, the system might not have started, and when it starts the formula must hold.}
% \drafter{To address the problem caused by the combination of stutter-steps and the next operation, another flag \texttt{f} was added.}
% \drafter{This flag is inverted on every transition of the simluated transition system.}
% \drafter{This flag will be the state in two successive steps if and only if the step was a stutter.}
% \drafter{With this rewrite, the LTL formulas are encoded in the following way ($\texttt{moved} \triangleq \texttt{f} \neq \mathsf{X}~\texttt{f}$).}

\begin{enumerate}[label=\roman*.]
  \item {\neg~\texttt{started}~\mathsf{U}~(\texttt{started} \wedge \lozenge~\square~$c$)}

  \item {\neg~\texttt{started}~\mathsf{U}~(\texttt{started} \wedge \square~\lozenge~$c$)}

  \item {\neg~\texttt{started}~\mathsf{U}~(\texttt{started} \wedge ((\mathsf{X}~\neg~$c$ ~\wedge~\texttt{moved}~\wedge~\mathsf{X}~\texttt{moved}) \rightarrow (\mathsf{X}~\mathsf{X}~$c$)))}

  \item {\neg~\texttt{started}~\mathsf{U}~(\texttt{started} \wedge \square~$a$)}

  \item {\neg~\texttt{started}~\mathsf{U}~(\texttt{started} \wedge ($a$~\mathsf{U}~\square~($b$ \vee $c$))))}

\end{enumerate}

\drafter{By using this encoding of the LTL formulas, Spin confirms the results from the previous exercise.}
\drafter{The counter-example provided are the following\footnote{The complete output can be found in files \texttt{extras/spin-output-1.1.b.*.txt}}.}

\begin{enumerate}[label=\roman*.]
  \item $s_1 s_3 s_4 s_2 s_4 (s_2 s_4)^\omega$
  \addtocounter{enumi}{2}
  \item $s_1 s_3 s_4 s_2 s_4 s_2 s_4 s_3 s_4$
\end{enumerate}

\subsection*{Exercise 1.2}

\drafter{Given that the exercise asks to state the equivalence or not of an LTL and a CTL formula, a definition and a theorem are restated for convenience \footnote{These are copied from Module 7 of the lecture notes}.}

\drafter{\textbf{Definition} Let $\varphi$ be an LTL formula and $\alpha$ be a CTL formula (both over $\Xi$). Then, $\varphi$ and $\alpha$ are said to be equivalent, denoted by $\varphi \equiv \alpha$, if $T \models \varphi$ if and only if $T \models \alpha$.}

\drafter{\textbf{Theorem} Let $\alpha$ be a CTL formula and let $\varphi$ be the LTL formula obtained from $\alpha$ by omitting all path quantifiers. Then, $\varphi \equiv \alpha$ or there is no LTL formula that is equivalent to $\alpha$}

\begin{enumerate}[label=(\alph*)]
  \item Let $\alpha = \mathsf{A}~\mathsf{G}~\mathsf{A}~\mathsf{X}~p$ and $\varphi = \mathsf{G}~\mathsf{X}~p$.

    $\mathsf{A}~\mathsf{G}~\mathsf{A}~\mathsf{X}~p$ states that $\mathsf{A}~\mathsf{X}~p$ is an invariants, i.e., holds for every state in every path.
    $\mathsf{A}~\mathsf{X}~p$ states that $p$ holds in all successors.
    For this reason $\alpha$ states that $p$ holds in all states after the first.

    $\varphi$ states that $p$ holds everywhere but the first state (it might or not hold in the first state).

  \item Let $\alpha = \mathsf{A}~\mathsf{F}~\mathsf{A}~\mathsf{X}~p$ and $\varphi = \mathsf{F}~\mathsf{X}~p$.
    % The formula corresponding to $\alpha$ is $\varphi = \mathsf{F}~\mathsf{X}~p$.
    $\mathsf{A}~\mathsf{F}~\mathsf{A}~\mathsf{X}~p$ states that in all paths eventually a state will be such where all sucessors satisfy $p$.
    A transition system satisfies $\mathsf{F}~\mathsf{X}~p$ if all its traces eventually have a states that is not initial that satisfies $p$.
    These are not equivalent and proved by the following counter-example, which satisfies $\varphi$ but not $\alpha$.

    \begin{figure}[ht]
      \centering
      \begin{tikzpicture}
        \node[state, initial,label=below:{\{\}}] (q1) {$s_0$};
        \node[state, right of=q1,label=below:{\{\}}] (q2) {$s_1$};
        \node[state, right of=q2,label=below:{\{$p$\}}] (q3) {$s_2$};
        \node[state, right of=q3,label=below:{\{\}}] (q4) {$s_3$};
        \node[state, right of=q4,label=below:{\{\}}] (q5) {$s_4$};

        \draw (q1) edge[] (q2)
              (q2) edge[] (q3)
              (q3) edge[] (q4)
              (q4) edge[] (q5)
              (q5) edge[bend right] (q3)
              (q2) edge[bend right=40] (q4)
      \end{tikzpicture}
      \caption{Counter-example for exercise 1.2 (b)}
      \label{fig:1-2-b}
    \end{figure}

  % TODO:
  \item Let $\alpha = \mathsf{A}~\mathsf{F}~p \vee \mathsf{A}~\mathsf{F}~q$ and $\varphi = \mathsf{F}~(p \vee q)$.
    The formula corresponding to $\alpha$ without the path quantifiers is $\psi = \mathsf{F}~p \vee \mathsf{F}~q$.
    $psi$ and $\alpha$ are not equivalent as demonstrated by the following transition system that satisfies the former, but not the latter:

    \begin{figure}[ht]
      \centering
      \begin{tikzpicture}
        \node[state, initial,label=below:{\{\}}] (q1) {$s_0$};
        \node[state, right of=q1, label=below:{\{$p$\}}] (q2) {$s_1$};
        \node[state, right of=q2, label=below:{\{$q$\}}] (q3) {$s_2$};

        \draw (q1) edge[] (q2)
              (q1) edge[bend left] (q3)
              (q2) edge[loop right] (q2)
              (q3) edge[loop above] (q3)
      \end{tikzpicture}
      \caption{Counter-example for exercise 1.2 (c)}
      \label{fig:1-2-c}
    \end{figure}

  Given that $\alpha$ is not equivalent to $\psi$, there's no LTL formula equivalent to $\alpha$ by the previous theorem.
  So $\alpha$ and $\varphi$ are not equivalent.

\end{enumerate}

\section*{Exercise 2}

\subsection*{Exercise 2.1}

\drafter{The encoding of properties provided is the following:}

\begin{enumerate}[label=(\alph*)]
  \item $\mathsf{G}~\neg(\texttt{W\_LIGHT} \wedge \texttt{E\_LIGHT})$

  \item $\mathsf{G}(\texttt{W\_LIGHT} \rightarrow \mathsf{F}~\neg\texttt{W\_LIGHT} \wedge \texttt{E\_LIGHT} \rightarrow \mathsf{F}~\neg\texttt{E\_LIGHT})$

  \item $\mathsf{G}(\texttt{on\_lane} \rightarrow \mathsf{F}~\neg\texttt{on\_lane})$

  \item $\mathsf{G}((\texttt{W\_LIGHT} \wedge \texttt{on\_lane}) \rightarrow (\neg\texttt{E\_LIGHT}~\mathsf{W}~\neg\texttt{on\_lane}))$

  \item $\mathsf{G}(\texttt{at\_w} \rightarrow \mathsf{F}~\texttt{W\_LIGHT})$
\end{enumerate}

\subsection*{Exercise 2.2}

\drafter{For convinience the definition of safety and liveness properties are restated (this is taken from Module 4 of the lecture notes):}

\drafter{\textbf{Definition} An LT property $P_{safe}$ over $\Xi$ is called a safety property if for all words $\sigma \in (2^\Xi)^\omega \ P_{safe}$ there exists a finite prefix $\hat{\sigma}$ of $\sigma$ such that $P_{safe} \cap \{\sigma' \in (2^\Xi)^\omega~|~\hat{\sigma}$ is a finite prefix of $\sigma' \} = \emptyset $}.

\drafter{\textbf{Definition} An LT property $P_{live}$ over $\Xi$ is called a liveness property if $pref(P_{live}) = (2^{\Xi})^*$}.

\drafter{Put simple, a property is a safety property if all words that are not in the language are not in the language because of a prefix -- any other word with that same prefix won't be in the language either (because in that prefix something bad happened) and a liveness property is one for each all finite prefixes are prefixes of some word in the language (become something good might still happen).}

\drafter{With the definitions just presented, the formulas can be classified in the following manner.}

\begin{enumerate}[label=(\alph*)]
  \item \drafter{Safety property: a trace where this is violated is a trace where two traffic lights are green at the same time. This is a finite trace and any trace for which this is a prefix won't be in the language.}

  \item \drafter{Liveness property: a finite trace where the traffic-light is green can always be extended to a good trace by making the light go red.}

  \item \drafter{Liveness property: a finite trace where there are cars on the shared lane can be extended to a good trace by making the cars leave the shared lane.}

  \item \drafter{Safety property: this property is violated by having an instant where the west traffic-light is green and there's a car in the shared lane and still the east traffic-light is also green. Any word not in the language must have such a bad prefix and any word with such a bad prefix won't be in the language.}

  \item \drafter{Liveness property: a trace that is not in this language is one where there was at some point a car waiting and the west traffic-light never turned green. Any prefix of this word is such that it can be extended so that a word in the language is obtained -- by turning the light green.}

\end{enumerate}

\subsection*{Exercise 2.3}

\drafter{As requested, the LTL formulas were written to \texttt{road\_ltl.pml} and Spin was used to check if model satisfies properties -- Spin's output of running the model can be found at \texttt{extras/spin\allowbreak-output-2.3.*.txt}.}
\drafter{The (very simplified) counter example provided by Spin was the following\footnote{The complete example is available in the \texttt{extras}}:}

\begin{enumerate}
  \item Car at east
  \item Car at west
  \item Car to shared
  \item Car exits shared
  \item In loop:
  \begin{enumerate}[label=\theenumi.\arabic*.]
    \item Car at east
    \item Car to shared
    \item Car exits shared
  \end{enumerate}
\end{enumerate}

\subsection*{Exercise 2.4}

\drafter{The initial model fails to satisfy property (e) because it does truly ensure fairness.}
\drafter{Weak fairness relies on an action being continuously enabled, but that does not happen.}
\drafter{If one car crosses, the safety properties prevent the other from crossing and safety disables it.}
\drafter{So a stronger notion of safety is required -- that if west just passed, it can only go if east it not waiting.}
\drafter{This can be done by having a single flag for this.}

\section*{Exercise 3}

\subsection*{Exercise 3.1}

% TODO: explain main modelling (particularly the main tricks on byz)

\subsection*{Exercise 3.2}

\drafter{The protocol should satisfy three properties: agreement, validity and termination.}
\drafter{For the purposes of specifying the formulas, let $inputs_i \in \{0,1\}$ denote the inputs of process $P_i for i \in \{1, .., |N|\}$ and $outputs_i \in \{0,1, \bot\}$ denote the output ($\bot$ is process hasn't outputed yet).}
\drafter{Note that this is only defined for correct processes.}
\drafter{For convenience let $done \triangleq \bigwedge_{i \in \{1, .., |N|\}} outputs_i \neq \bot$, given an array $v$, $allSame(v) \triangleq \exists x \, \forall y \in v \, (x = y)$.}
\drafter{The encoding of these properties as LTL formulas is the following.}

\begin{itemize}
  \item Agreement: $\square (done \rightarrow allSame(outputs))$
  \item Validity: $\square ((done \wedge allSame(inputs)) \rightarrow allSame(outputs))$
  \item Termination: $\lozenge~done$
\end{itemize}

\drafter{We noted that the validity property that the project statement specified was different from what the paper specified and from what is Distributed System's usual notion of validity.}
\drafter{The previous definition refered to the latter.}
\drafter{For completeness the former, which requires the intermediate values to be uniform throughput the execution, is specified (we call it strong validity):}
\drafter{For this we need to assume a common vector $v$ where $v_i$ is the value of process $i$.}

\begin{itemize}
  \item Strong validity: $\square (allSame(inputs) \rightarrow (allSame(v) \wedge (done \rightarrow allSame(outputs))))$
\end{itemize}

\subsection*{Exercise 3.3}

\drafter{As requested, Spin's output of running the model can be found at \texttt{extras/spin-output-3.3.*.txt}.}

\subsection*{Exercise 3.4}

\drafter{To check that this condition is indeed required, the number of faulty nodes was increase.}
\drafter{A single extra faulty process was added by setting \texttt{T} to 2.}

% TODO: hand output

\end{document}
